\documentclass{beamer}

\newcommand{\lesson}{Inheritance}

\author[Chris Simpkins]
{Christopher Simpkins \\\texttt{chris.simpkins@gatech.edu}}
\institute[Georgia Tech] % (optional, but mostly needed)

\date{}


\newcommand{\course}{Introduction to Object-Oriented Programming}
\subject{\course}
\title[\lesson]{\course}
\subtitle{\lesson}

\author[CS 1331]
{Christopher Simpkins \\\texttt{chris.simpkins@gatech.edu}}
\institute[Georgia Tech]

\date[]{}

\newcommand{\link}[2]{\href{#1}{\textcolor{blue}{\underline{#2}}}}
\newcommand{\code}{cs1331.org/code}

\usepackage{colortbl}

% If you have a file called "university-logo-filename.xxx", where xxx
% is a graphic format that can be processed by latex or pdflatex,
% resp., then you can add a logo as follows:

% \pgfdeclareimage[width=0.6in]{coc-logo}{cc_2012_logo}
% \logo{\pgfuseimage{coc-logo}}

\mode<presentation>
{
  \usetheme{Berlin}
  \useoutertheme{infolines}

  % or ...

 \setbeamercovered{transparent}
  % or whatever (possibly just delete it)
}

\usepackage{tikz}
% Optional PGF libraries
\usepackage{pgflibraryarrows}
\usepackage{pgflibrarysnakes}
\usepackage{pgfplots}
\usepackage{fancybox}
\usepackage{listings}
\usepackage{hyperref}
\hypersetup{colorlinks=true,urlcolor=blue}
\usepackage[english]{babel}
% or whatever

\usepackage[latin1]{inputenc}
% or whatever

\usepackage{times}
\usepackage[T1]{fontenc}
% Or whatever. Note that the encoding and the font should match. If T1
% does not look nice, try deleting the line with the fontenc.


\usepackage{listings}

% "define" Scala
\lstdefinelanguage{scala}{
  morekeywords={abstract,case,catch,class,def,%
    do,else,extends,false,final,finally,%
    for,if,implicit,import,match,mixin,%
    new,null,object,override,package,%
    private,protected,requires,return,sealed,%
    super,this,throw,trait,true,try,%
    type,val,var,while,with,yield},
  otherkeywords={=>,<-,<\%,<:,>:,\#,@},
  sensitive=true,
  morecomment=[l]{//},
  morecomment=[n]{/*}{*/},
  morestring=[b]",
  morestring=[b]',
  morestring=[b]""",
}

\usepackage{color}
\definecolor{dkgreen}{rgb}{0,0.6,0}
\definecolor{gray}{rgb}{0.5,0.5,0.5}
\definecolor{mauve}{rgb}{0.58,0,0.82}

% Default settings for code listings
\lstset{frame=tb,
  language=scala,
  aboveskip=2mm,
  belowskip=2mm,
  showstringspaces=false,
  columns=flexible,
  basicstyle={\scriptsize\ttfamily},
  numbers=none,
  numberstyle=\tiny\color{gray},
  keywordstyle=\color{blue},
  commentstyle=\color{dkgreen},
  stringstyle=\color{mauve},
  frame=single,
  breaklines=true,
  breakatwhitespace=true,
  keepspaces=true
  %tabsize=3
}


% \beamerdefaultoverlayspecification{<+->}


\begin{document}

\begin{frame}
  \titlepage
\end{frame}

%------------------------------------------------------------------------
\begin{frame}[fragile]{Programming in the Large}


Software is complex.  Three ways we deal with complexity:
\begin{itemize}
\item Abstraction - boiling a concept down to its essential elements, ignoring irrelevant details
\item Decomposition - decompose system into packages, classes, functions
\item Reuse - reuse library function in many diferent places
\end{itemize}
\vspace{.1in}
Today we introduce another kind of resuse: inheritance

\end{frame}
%------------------------------------------------------------------------

%------------------------------------------------------------------------
\begin{frame}[fragile]{What is inheritance?}

\begin{center}
\includegraphics[height=2.5in]{money_Inheritance.jpg}\footnote{Source: http://talentenbank.com/can-you-really-make-inheritance-into-a-good-financial-move-in-the-long-run}
\end{center}

\end{frame}
%------------------------------------------------------------------------

%------------------------------------------------------------------------
\begin{frame}[fragile]{What is inheritance?}


More like genetics ...
\begin{center}
\includegraphics[height=2in]{puppy-inheritance.jpg}\footnote{Source: http://www.dnaftb.org/5/}\\
\end{center}
... but a programming concept that, like so much in CS, borrows a term from another field to leverage our intuition.

\end{frame}
%------------------------------------------------------------------------

%------------------------------------------------------------------------
\begin{frame}[fragile]{Inheritance}


Inheritance:  deriving one class from another class.
\begin{lstlisting}[language=Java]
public class Employee { ... }
public class HourlyEmployee extends Employee { ... }
public class SalariedEmployee extends Employee { ... }
\end{lstlisting}

\begin{itemize}
\item {\tt Employee} is the {\it base class} or {\it superclass}
\item {\tt HourlyEmployee} and {\tt SalariedEmployee} are {\it derived classes} or {\it subclasses}
\item Subclasses {\it inherit} the interface and implementation of their superclass(es)
\item {\tt extends} is the Java syntax for inheriting from another class
\end{itemize}

Important idea to plant in your head now: subclassing is about concept reuse not merely implementation reuse.  For example, {\tt HourlyEmployee} {\it is-a} {\tt Employee} conceptually.

\end{frame}
%------------------------------------------------------------------------

%------------------------------------------------------------------------
\begin{frame}[fragile]{Superclasses}

\vspace{-.05in}
Consider the superclass \href{\code/employee/Employee1.java}{Employee1}:\footnote{Note that we'll number the versions of our Employee classes like we did with Card.}
\vspace{-.05in}
\begin{lstlisting}[language=Java]
public class Employee1 {
    private String name;
    private Date hireDate;

    public Employee1(String aName, Date aHireDate) {
        disallowNullArguments(aName, aHireDate);
        name = aName;
        hireDate = aHireDate;
    }
    public String getName() {
        return name;
    }
    public Date getHireDate() {
        return hireDate;
    } // and toString(), etc. ...
}
\end{lstlisting}
\vspace{-.05in}
{\tt Employee} defines the basic information needed to define any employee.

\end{frame}
%------------------------------------------------------------------------

%------------------------------------------------------------------------
\begin{frame}[fragile]{Subclasses}


The {\tt extends} clause names the direct superclass of the current class (\href{http://docs.oracle.com/javase/specs/jls/se7/html/jls-8.html#jls-8.1.4)}{JLS \S 8.1.4}).

Here is a subclass of {\tt Employee1},  \link{\code/employee/HourlyEmployee1.java}{HourlyEmployee1}:
\begin{lstlisting}[language=Java]
public class HourlyEmployee extends Employee {

    public HourlyEmployee(String aName, Date aHireDate) {
        super(aName, aHireDate);
    }
}
\end{lstlisting}

\begin{itemize}
\item {\tt HourlyEmployee} inherits all the members of {\tt Employee}
\item {\tt HourlyEmployee} can't access private members of {\tt Employee} directly
\item The {\tt super} call in the constructor calls {\tt Employee}'s constructor to initialize {\tt HourlyEmployee} instances
\end{itemize}
The {\tt HourlyEmployee} concept extends the {\tt Employee} concept.

\end{frame}
%------------------------------------------------------------------------

%------------------------------------------------------------------------
\begin{frame}[fragile]{{\tt super} Subtleties}


\begin{itemize}
\item If present, an explicit {\tt super} call must be the first statement in a constructor.
\item If an explicit {\tt super} call is not present and the superclass has a no-arg constructor, {\tt super()} will implicitly be the first statement in any constructor
\item If there is no no-arg constructor in a superclass (for example, if the superclass defines other constructors without explicitly defining a no-arg constructor), then subclass constructors must explicitly include a {\tt super} call.
\end{itemize}
Together, these rules enforce an ``inside-out'' construction order for objects: the highest superclass piece of an object is initialzed first, followed by the second highest, and so on.

\end{frame}
%------------------------------------------------------------------------

%------------------------------------------------------------------------
\begin{frame}[fragile]{Subclass Constructors}


Recall our definitions of {\tt Employee1} and {\tt HourlyEmployee1}.
\begin{lstlisting}[language=Java]
public class Employee1 {
    // The only constructor in Employee
    public Employee1(String aName, Date aHireDate) {
        disallowNullArguments(aName, aHireDate);
        name = aName;
        hireDate = aHireDate;
    }
    // ...
}
\end{lstlisting}

\begin{lstlisting}[language=Java]
public class HourlyEmployee1 extends Employee1 {

    public HourlyEmployee1(String aName, Date aHireDate) {
        super(aName, aHireDate);
    }
}
\end{lstlisting}

Would {\tt HourlyEmployee1.java} compile if we left off the constructor definition?

\end{frame}
%------------------------------------------------------------------------


%------------------------------------------------------------------------
\begin{frame}[fragile]{Inherited Members}


Given our previous definitions of {\tt Employee1} and {\tt HourlyEmployee1}, we can write code like this (from \link{\code/employee/EmployeeDemo1.java}{EmployeeDemo1}):
\begin{lstlisting}[language=Java]
DateFormat df = DateFormat.getDateInstance();
HourlyEmployee eva = new HourlyEmployee("Eva L. Uator",
                                        df.parse("February 18, 2013"));
System.out.println(eva.getName() + " was hired on "
                   + eva.getHireDate());
\end{lstlisting}
Note that
\begin{itemize}
\item we didn't have to define {\tt getName} and {\tt getHireDate} in {\tt HourlyEmployee}
\item our current implementation of {\tt HourlyEmployee} doesn't add anything to {\tt Employee}
\end{itemize}


\end{frame}
%------------------------------------------------------------------------

%------------------------------------------------------------------------
\begin{frame}[fragile]{Subclasses Specialize Superclasses}


We define subclasses to {\it extend} or {\it specialize} the functionality of their superclasses.  Let's add suitable extensions to {\tt HourlyEmployee}:\footnote{Employee2 is the same as Employee1, but we'll keep the numbers consistent to avoid confusion.}
\vspace{-.05in}
\begin{lstlisting}[language=Java]
public class HourlyEmployee2 extends Employee2 {
    private double hourlyWage;
    private double monthlyHours;

    public HourlyEmployee(String aName, Date aHireDate,
                          double anHourlyWage, double aMonthlyHours) {
        super(aName, aHireDate);
        disallowZeroesAndNegatives(anHourlyWage, aMonthlyHours);
        hourlyWage = anHourlyWage;
        monthlyHours = aMonthlyHours;
    }
    public double getHourlyWage() { return hourlyWage;}
    public double getMonthlyHours() { return monthlyHours;}
    public double getMonthlyPay() { return hourlyWage * monthlyHours; }
    // ...
}
\end{lstlisting}
\vspace{-.1in}
Food for thought: what is the monthly pay rule for {\tt HourlyEmployee}s?  What if an employee works more than 40 hours per week?
\end{frame}
%------------------------------------------------------------------------

%------------------------------------------------------------------------
\begin{frame}[fragile]{Access Modifiers}


\begin{center}
\begin{tabular}{|l|c|c|c|c|} \hline
Modifier & Class & Package & Subclass & World\\
\hline
public & Y & Y & Y & Y\\
protected & Y & Y & Y & N\\
no modifier & Y & Y & N & N\\
private & Y & N & N & N\\
\hline
\end{tabular}
\end{center}

\begin{itemize}
\item Every class has an access level (for now all of our classes are {\tt public}).
\item Every member has an access level.
\item The defulat access level, no mofifier, is also called ``package private.''
\end{itemize}

\end{frame}
%------------------------------------------------------------------------

%------------------------------------------------------------------------
\begin{frame}[fragile]{Access Restrictions Extend to Subclasses}


{\tt private} members of superclasses are present in subclasses, but can't be directly accessed.  So this won't compile:
\vspace{-.05in}
\begin{lstlisting}[language=Java]
public class HourlyEmployee2 extends Employee2 {
  // ...
  public String toString() {
    return name + "; Hire Date: " + hireDate + "; Hourly Wage: "
    + hourlyWage + "; Monthly Hours: " + monthlyHours;
  }
}
\end{lstlisting}
because {\tt name} and {\tt hireDate} are private in {\tt Employee}.  But their getter methods are public:
\vspace{-.05in}
\begin{lstlisting}[language=Java]
public class HourlyEmployee2 extends Employee2 {
  // ...
  public String toString() {
    return getName()+"; Hire Date: "+getHireDate() +"; Hourly Wage: "
    + hourlyWage + "; Monthly Hours: " + monthlyHours;
  }
}
\end{lstlisting}


\end{frame}
%------------------------------------------------------------------------

%------------------------------------------------------------------------
\begin{frame}[fragile]{Overriding Methods}

\vspace{-.05in}
Overriding a method means providing a new definition of a superclass method in a subclass.  We've been doing this all along with {\tt toString} and {\tt equals}, which are defined in {\tt java.lang.Object}, the highest superclass of all Java classes.
\vspace{-.05in}
\begin{lstlisting}[language=Java]
public class Object {
    public String toString() {
        return getClass().getName() + "@"
            + Integer.toHexString(hashCode());
    }
    public boolean equals(Object obj) {
        return (this == obj);
    }
}
\end{lstlisting}
\vspace{-.1in}
We redefine these on our classes because
\begin{itemize}
\item the default implementation of {\tt toString} just prints the class name and hash code (which is the memory address by default).
\item the default implementation of {\tt equals} just compares object references, i.e., identity equality, when what we want from {\tt equals} is value equality
\end{itemize}


\end{frame}
%------------------------------------------------------------------------

%------------------------------------------------------------------------
\begin{frame}[fragile]{{\tt @Override} Annotation}

The optional {\tt @Override} \link{http://docs.oracle.com/javase/tutorial/java/annotations/index.html}{annotation} informs the compiler that the element is meant to override an element declared in a superclass.

\begin{lstlisting}[language=Java]
public class Employee2 {
  // ...
  @Override
  public String toString() {
    return name + "; Hire Date: " + hireDate;
  }
}
\end{lstlisting}
Now if our subclass's {\tt toString()} method doesn't actually override {\tt Java.lang.Object}'s (or some other class's) {\tt toString()}, the compiler will tell us.
\end{frame}
%------------------------------------------------------------------------

%% %------------------------------------------------------------------------
%% \begin{frame}[fragile]{Mutating Objects}


%% Recall our deifinition of {\tt HourlyEmployee}:
%% \vspace{-.05in}
%% \begin{lstlisting}[language=Java]
%% public class HourlyEmployee extends Employee {
%%     private double hourlyWage;
%%     private double monthlyHours;

%%     public HourlyEmployee(String aName, Date aHireDate,
%%                           double anHourlyWage, double aMonthlyHours) {
%%         super(aName, aHireDate);
%%         disallowZeroesAndNegatives(anHourlyWage, aMonthlyHours);
%%         hourlyWage = anHourlyWage;
%%         monthlyHours = aMonthlyHours;
%%     }
%%     public double getHourlyWage() { return hourlyWage;}
%%     public double getMonthlyHours() { return monthlyHours;}
%%     public double getMonthlyPay() { return hourlyWage * monthlyHours; }
%%     // ...
%% }
%% \end{lstlisting}
%% \vspace{-.1in}
%% \begin{itemize}
%% \item What if we wanted to give an employee a raise?  More or fewer hours?
%% \end{itemize}

%% \end{frame}
%% %------------------------------------------------------------------------


%% %------------------------------------------------------------------------
%% \begin{frame}[fragile]{Copy Constructors}


%% A copy constructor simplifies new object construction.  Here are the copy constructors for {\tt Emplyoyee}
%% \vspace{-.05in}
%% \begin{lstlisting}[language=Java]
%% public class Employee {
%%     public Employee(Employee other) {
%%         this.name = other.name;
%%         this.hireDate = other.hireDate;
%%     } // ...
%% }
%% \end{lstlisting}

%% and {\tt HourlyEmployee}
%% \vspace{-.05in}
%% \begin{lstlisting}[language=Java]
%% public class HourlyEmployee extends Employee {
%%    public HourlyEmployee(HourlyEmployee other) {
%%         super(other);
%%         this.hourlyWage = other.hourlyWage;
%%         this.monthlyHours = other.monthlyHours;
%%     } // ...
%% }
%% \end{lstlisting}

%% Can you think of an important consideration in writing copy constructors?

%% \end{frame}
%% %------------------------------------------------------------------------


%------------------------------------------------------------------------
\begin{frame}[fragile]{Explicit Constructor Invocation with {\tt this}}


What if we wanted to have default default values for hourly wages and monthly hours?  We can provide an alternate constructor that delegates to our main constructor with {\tt this} \link{\code/employee/HourlyEmployee3.java}{HourlyEmployee3.java}:
\begin{lstlisting}[language=Java]
public final class HourlyEmployee3 extends Employee3 {
    /**
     * Constructs an HourlyEmployee with hourly wage of 20 and
     * monthly hours of 160.
     */
    public HourlyEmployee3(String aName, Date aHireDate) {
        this(aName, aHireDate, 20.00, 160.0);
    }
    public HourlyEmployee3(String aName, Date aHireDate,
                          double anHourlyWage, double aMonthlyHours) {
        super(aName, aHireDate);
        disallowZeroesAndNegatives(anHourlyWage, aMonthlyHours);
        hourlyWage = anHourlyWage;
        monthlyHours = aMonthlyHours;
    }
    // ...
}
\end{lstlisting}

\end{frame}
%------------------------------------------------------------------------

%------------------------------------------------------------------------
\begin{frame}[fragile]{{\tt this} and {\tt super}}


\begin{itemize}
\item If present, an explicit constructor call must be the first statement in the constructor.
\item Can't have both a {\tt super} and {\tt this} call in a constructor.
\item A constructor with a {\tt this} call must call, either directly or indirectly, a constructor with a {\tt super} call (implicit or explicit).
\end{itemize}

\begin{lstlisting}[language=Java]
public final class HourlyEmployee3 extends Employee3 {
    public HourlyEmployee3(String aName, Date aHireDate) {
        this(aName, aHireDate, 20.00);
    }
    public HourlyEmployee3(String aName, Date aHireDate, double anHourlyWage) {
        this(aName, aHireDate, anHourlyWage, 160.0);
    }
    public HourlyEmployee3(String aName, Date aHireDate,
                          double anHourlyWage, double aMonthlyHours) {
        super(aName, aHireDate);
        disallowZeroesAndNegatives(anHourlyWage, aMonthlyHours);
        hourlyWage = anHourlyWage;
        monthlyHours = aMonthlyHours;
    }
    // ...
}
\end{lstlisting}


\end{frame}
%------------------------------------------------------------------------

%------------------------------------------------------------------------
\begin{frame}[fragile]{The Liskov Substitution Principle (LSP)}

\begin{quote}
Subtypes must be substitutable for their supertypes.
\end{quote}
Consider the method:
\begin{lstlisting}[language=Java]
    public static Date vestDate(Employee employee) {
        Date hireDate = employee.getHireDate();
        int vestYear = hireDate.getYear() + 2;
        return new Date(vestYear,
                        hireDate.getMonth(),
                        hireDate.getDay());
    }
\end{lstlisting}
We can pass any subtype of {\tt Employee} to this method:
\begin{lstlisting}[language=Java]
        DateFormat df = DateFormat.getDateInstance();
        HourlyEmployee eva = new HourlyEmployee("Eva L. Uator",
                           df.parse("February 13, 2013"), 20.00, 200);
        Date evaVestDate = vestDate(eva);
\end{lstlisting}

We must ensure that subtypes are indeed substitutable for supertypes.

\end{frame}
%------------------------------------------------------------------------

%------------------------------------------------------------------------
\begin{frame}[fragile]{LSP Counterexample}

A suprising counter-example:
\vspace{-.05in}
\begin{lstlisting}[language=Java]
public class Rectangle {
  public void setWidth(double w) { ... }
  public void setHeight(double h) { ... }
}
public class Square extends Rectangle {
  public void setWidth(double w) {
    super.setWidth(w);
    super.setHeight(w);
  }
  public void setHeight(double h) {
    super.setWidth(h);
    super.setHeight(h);
  }
}
\end{lstlisting}
\vspace{-.1in}
\begin{itemize}
\item We know from math class that a square ``is a'' rectangle.
\item The overridden {\tt setWidth} and {\tt setHeight} methods in {\tt Square} enforce the class invariant of {\tt Square}, namely, that {\tt width == height}.
\end{itemize}


\end{frame}
%------------------------------------------------------------------------

%------------------------------------------------------------------------
\begin{frame}[fragile]{LSP Violation}


Consider this client of {\tt Rectangle}:
\begin{lstlisting}[language=Java]
public void g(Rectangle r) {
  r.setWidth(5);
  r.setHeight(4);
  assert r.area() == 20;
}
\end{lstlisting}

\begin{itemize}
\item Client (author of {\tt g}) assumes width and height are independent in {\tt r} becuase {\tt r} is a {\tt Rectangle}.
\item If the {\tt r} passed to {\tt g} is actually an instance of {\tt Square}, what will be the value of {\tt r.area()}?
\end{itemize}
The Object-oriented {\tt is-a} relationship is about behavior.  {\tt Square}'s {\tt setWidth} and {\tt setHeight} methods don't behave the way a {\tt Rectangle}'s {\tt setWidth} and {\tt setHeight} methods are expected to behave, so a {\tt Square} doesn't fit the object-oriented {\it is-a} {\tt Rectangle} definition.  Let's make this more formal ...

\end{frame}
%------------------------------------------------------------------------

%------------------------------------------------------------------------
\begin{frame}[fragile]{Conforming to LSP: Design by Contract}


\begin{quote}
Require no more, promise no less.
\end{quote}

Author of a class specifies the behavior of each method in terms of preconditions and postconditions.  Subclasses must follow two rules:
\begin{itemize}
\item Preconditions of overriden methods must be equal to or weaker than those of the superclass (enforces or assumes no more than the constraints of the superclass method).
\item Postconditions of overriden methods must be equal to or greater than those of the superclass (enforces all of the constraints of the superclass method and possibly more).
\end{itemize}

In the Rectangle-Square case the postcondition of {\tt Rectangle}'s {\tt setWidth} method:
\begin{lstlisting}
assert((rectangle.w == w) && (rectangle.height == old.height))
\end{lstlisting}
cannot be satisfied by {\tt Square}, which tells us that a {\tt Square} doesn't satisfy the object-oriented {\it is-a} relationship to {\tt Rectangle}.

\end{frame}
%------------------------------------------------------------------------

%------------------------------------------------------------------------
\begin{frame}[fragile]{LSP Conforming 2D Shapes}

\begin{lstlisting}[language=Java]
public interface 2dShape {
    double area();
}
public class Rectangle implements 2dShape {
    public void setWidth(double w) { ... }
    public void setHeight(double h) { ... }
    public double area() {
        return width * height;
    }
}
public class Square implements 2dShape {
    public void setSide(double w) { ... }
    public double area() {
        return side * side;
    }
}
\end{lstlisting}

Notice the use of an \link{http://docs.oracle.com/javase/tutorial/java/IandI/createinterface.html}{interface} to define a type.

\end{frame}
%------------------------------------------------------------------------

%------------------------------------------------------------------------
\begin{frame}[fragile]{Interfaces}


An interface represents an object-oriented type: a set of public methods (declarations, not definitions) that any object of the type supports.  Recall the {\tt 2dShape} interface:

\begin{lstlisting}[language=Java]
public interface 2dShape {
    double area();
}
\end{lstlisting}

You can't instantiate interfaces.  So you must define a class that implements the interface in order to use it.  Implementing an interface is similar to extending a class, but uses the {\tt implements} keyword:

\begin{lstlisting}[language=Java]
public class Square implements 2dShape {
    public void setSide(double w) { ... }
    public double area() {
        return side * side;
    }
}
\end{lstlisting}

Now a {\tt Square} {\it is-a} {\tt 2dShape}.


\end{frame}
%------------------------------------------------------------------------


%------------------------------------------------------------------------
\begin{frame}[fragile]{Interfaces Define a Type}
\vspace{-.05in}
\begin{lstlisting}[language=Java]
public interface 2dShape {
    double area();
}
\end{lstlisting}
\vspace{-.05in}
This means that any object of type {\tt 2dShape} supports the {\tt area} method, so we can write code like this:
\vspace{-.05in}
\begin{lstlisting}[language=Java]
public double calcTotalArea(2dShape ... shapes) {
    double area = 0.0;
    for (2dShape shape: shapes) {
        area += shape.area();
    }
    return area;
}
\end{lstlisting}
\vspace{-.05in}

Two kinds of inheritance: {\it implementation} and {\it interface} inheritance.

\begin{itemize}
\item extending a class means inheriting both the interface and the implementation of the superclass
\item implementing an interface means inheriting only the interface, that is, the public methods
\end{itemize}

\end{frame}
%------------------------------------------------------------------------

%% %------------------------------------------------------------------------
%% \begin{frame}[fragile]{Default Methods in Interfaces}


%% \begin{lstlisting}[language=Java]

%% \end{lstlisting}

%% \begin{itemize}
%% \item
%% \end{itemize}


%% \end{frame}
%% %------------------------------------------------------------------------

%% %------------------------------------------------------------------------
%% \begin{frame}[fragile]{Conflict Resolution for Default Methods}


%% \begin{lstlisting}[language=Java]

%% \end{lstlisting}

%% \begin{itemize}
%% \item Superclasses win.
%% \item Interfaces clash.
%% \end{itemize}


%% \end{frame}
%% %------------------------------------------------------------------------

%% %------------------------------------------------------------------------
%% \begin{frame}[fragile]{Static Methods in Interfaces}


%% \begin{lstlisting}[language=Java]

%% \end{lstlisting}

%% \begin{itemize}
%% \item
%% \end{itemize}


%% \end{frame}
%% %------------------------------------------------------------------------



% %------------------------------------------------------------------------
% \begin{frame}[fragile]{}


% \begin{lstlisting}[language=Java]

% \end{lstlisting}

% \begin{itemize}
% \item
% \end{itemize}


% \end{frame}
% %------------------------------------------------------------------------


%------------------------------------------------------------------------
\begin{frame}[fragile]{Programming Exercise}


To get some practice writing classes that use inheritance, write:
\begin{itemize}
\item A class named {\tt Animal} with:
\begin{itemize}
\item A private instance variable {\tt name}, with a public getter and setter. (Note: {\tt name} is a name of an animal, not the animal's species.)
\item A single constructor that takes the name of the {\tt Animal}
\item A public instance method {\tt speak} that returns a {\tt String} representation of the sound it makes.
\end{itemize}

\item A class named {\tt Dog} that extends {\tt Animal} and specializes the {\tt speak} method appropriately.

\item A {\tt Kennel} class with
\begin{itemize}
\item a private instance variable {\tt dogs} that is an array of {\tt Dog}
\item a single constructor that takes a variable number of single {\tt Dog} parameters and initializes the {\tt dogs} instance variable with the constructor's actual parameters.
\item a method {\tt soundOff()} that prints to {\tt STDOUT} ({\tt System.out}) one line for each {\tt Dog} in {\tt dogs} that reads ``[dog name] says [output of {\tt speak} method]!'', e.g. ``Chloe says woof, woof!''
\end{itemize}

\end{itemize}
We'll review this at the start of the next lecture.

\end{frame}
%------------------------------------------------------------------------

%% %------------------------------------------------------------------------
%% \begin{frame}[fragile]{}

%% \begin{lstlisting}[language=Java]
%% \end{lstlisting}

%% \begin{itemize}
%% \item
%% \end{itemize}

%% \end{frame}
%% %------------------------------------------------------------------------

\end{document}
