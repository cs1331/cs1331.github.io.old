\documentclass{beamer}

\mode<presentation>
{
  \usetheme{Berlin}
  \useoutertheme{infolines}

  % or ...

 \setbeamercovered{transparent}
  % or whatever (possibly just delete it)
}

\usepackage{hyperref}
\usepackage{fancybox}
\usepackage{listings}
\usepackage[abbr]{harvard}

\usepackage[english]{babel}
% or whatever

\usepackage[latin1]{inputenc}
% or whatever

\usepackage{times}
\usepackage[T1]{fontenc}
% Or whatever. Note that the encoding and the font should match. If T1
% does not look nice, try deleting the line with the fontenc.


\usepackage{listings}
 
% "define" Scala
\lstdefinelanguage{scala}{
  morekeywords={abstract,case,catch,class,def,%
    do,else,extends,false,final,finally,%
    for,if,implicit,import,match,mixin,%
    new,null,object,override,package,%
    private,protected,requires,return,sealed,%
    super,this,throw,trait,true,try,%
    type,val,var,while,with,yield},
  otherkeywords={=>,<-,<\%,<:,>:,\#,@},
  sensitive=true,
  morecomment=[l]{//},
  morecomment=[n]{/*}{*/},
  morestring=[b]",
  morestring=[b]',
  morestring=[b]""",
}

\usepackage{color}
\definecolor{dkgreen}{rgb}{0,0.6,0}
\definecolor{gray}{rgb}{0.5,0.5,0.5}
\definecolor{mauve}{rgb}{0.58,0,0.82}
 
% Default settings for code listings
\lstset{frame=tb,
  language=scala,
  aboveskip=3mm,
  belowskip=3mm,
  showstringspaces=false,
  columns=flexible,
  basicstyle={\scriptsize\ttfamily},
  numbers=none,
  numberstyle=\tiny\color{gray},
  keywordstyle=\color{blue},
  commentstyle=\color{dkgreen},
  stringstyle=\color{mauve},
  frame=single,
  breaklines=true,
  breakatwhitespace=true,
  keepspaces=true
  %tabsize=3
}


\title[CS 1331 Introduction to Object-Oriented Programming] % (optional, use only with long
                                      % paper titles)
{Strings}

\subtitle{}
%% {Include Only If Paper Has a Subtitle}

\author[Chris Simpkins] % (optional, use only with lots of authors)
{Christopher Simpkins \\\texttt{chris.simpkins@gatech.edu}}
% - Give the names in the same order as the appear in the paper.
% - Use the \inst{?} command only if the authors have different
%   affiliation.

\institute[Georgia Tech] % (optional, but mostly needed)

\date[] % (optional, should be abbreviation of
                           % conference name)
{}
% - Either use conference name or its abbreviation.
% - Not really informative to the audience, more for people (including
%   yourself) who are reading the slides online

\subject{Strings}
% This is only inserted into the PDF information catalog. Can be left
% out. 

% If you have a file called "university-logo-filename.xxx", where xxx
% is a graphic format that can be processed by latex or pdflatex,
% resp., then you can add a logo as follows:

%% \pgfdeclareimage[width=0.6in]{gtri_logo}{gtri_logo}
%% \logo{\pgfuseimage{gtri_logo}}

% Delete this, if you do not want the table of contents to pop up at
% the beginning of each subsection:
%% \AtBeginSection[]
%% {
%%   \begin{frame}<beamer>{Outline}

%%  \tableofcontents[currentsection]
%%   \end{frame}
%% }

% If you wish to uncover everything in a step-wise fashion, uncomment
% the following command: 

% \beamerdefaultoverlayspecification{<+->}


\begin{document}

\begin{frame}
  \titlepage
\end{frame}

%------------------------------------------------------------------------
\begin{frame}[fragile]{{\tt String} Values}


Two ways to represent {\tt String} values in a program:
\begin{itemize}

\item {\tt String} literals
\begin{lstlisting}[language=Java]
"foo"
\end{lstlisting}

\item {\tt String} variables

\begin{lstlisting}[language=Java]
String foo = "foo";
\end{lstlisting}

 
\end{itemize}
\end{frame}
%------------------------------------------------------------------------


%------------------------------------------------------------------------
\begin{frame}[fragile]{{\tt String} Concatenation }


The {\tt +} operator is overloaded to mean concatenation for {\tt String} objects

\begin{itemize}

\item Strings can be concatenated
\begin{lstlisting}[language=Java]
String bam = foo + bar + baz; // Now bam is "foobarbaz"
\end{lstlisting}

\item Primitive types can also be concatenated with {\tt Strings}.  The primitive is converted to a String
\begin{lstlisting}[language=Java]
String s = bam + 42; // s is "foobarbaz42"
String t = 42 + bam; // t is "42foobarbaz"
\end{lstlisting}

\end{itemize}

Note that {\tt +} is only overloaded for {\tt Strings}.

\end{frame}
%------------------------------------------------------------------------

%------------------------------------------------------------------------
\begin{frame}[fragile]{The {\tt String} Class}


{\tt String} acts like primitive thanks to syntactic sugar provided by the Java compiler, but it is defined as a class in the Java standard library

\begin{itemize}

\item See \url{http://docs.oracle.com/javase/7/docs/api/java/lang/String.html} for details.

\item Methods on objects are invoked on the object using the {\tt .} operator
\begin{lstlisting}[language=Java]
String empty = "";
int len = empty.length(); // len is 0
\end{lstlisting}

\item Look up the methods {\tt length}, {\tt indexOf}, {\tt substring}, and {\tt compareTo}, and {\tt trim}

\item Because {\tt String}s are objects, beware of null references:
\begin{lstlisting}[language=Java]
int aPosInBoom = boom.indexOf("a"); // This won't compile
\end{lstlisting}

\end{itemize}

\end{frame}
%------------------------------------------------------------------------

%------------------------------------------------------------------------
\begin{frame}[fragile]{Putting it all together}


Break out your laptops!

\end{frame}
%------------------------------------------------------------------------



\end{document}
