\documentclass[answers,addpoints,9pt]{exam}
\usepackage{verbatim, multicol, tabularx,graphicx}
\usepackage{amsmath,amsthm, amssymb, latexsym, listings}

\lstset{frame=tb,
  language=Java,
  aboveskip=1mm,
  belowskip=0mm,
  showstringspaces=false,
  columns=flexible,
  basicstyle={\ttfamily},
  numbers=none,
  frame=single,
  breaklines=true,
  breakatwhitespace=true
}


\title{CS 1331 Exam 3 Practice}
\date{}
\setcounter{page}{0}
\begin{document}

\maketitle
\thispagestyle{head}
\firstpageheader{}
              {\tiny Copyright \textcopyright\ 2014 All rights reserved. Duplication and/or usage for purposes of any kind without permission is strictly forbidden.}
              {}

\runningheader{}
              {\tiny Copyright \textcopyright\ 2014 All rights reserved. Duplication and/or usage for purposes of any kind without permission is strictly forbidden.}
              {}
\firstpagefooter{Page \thepage\ of \numpages}
              {}
              {Points available: \pointsonpage{ \thepage} -
               points lost: \makebox[.5in]{\hrulefill} =
               points earned:  \makebox[.5in]{\hrulefill}.
              Graded by: \makebox[.75in]{\hrulefill}}

\runningfooter{Page \thepage\ of \numpages}
              {}
              {Points available: \pointsonpage{ \thepage} -
               points lost: \makebox[.5in]{\hrulefill} =
               points earned:  \makebox[.5in]{\hrulefill}.
              Graded by: \makebox[.75in]{\hrulefill}}

\ifprintanswers
\begin{center}
{\LARGE ANSWER KEY}
\end{center}
\else
\vspace{0.1in}
\hbox to \textwidth{Name (print clearly): \enspace\hrulefill}
\vspace{0.3in}
\hbox to \textwidth{Signature: \hrulefill}

\vspace{0.3in}
\hbox to \textwidth{GT account username (gtg, gth, msmith3, etc): \enspace\hrulefill}

\fi

\vfill

\begin{itemize}
\item Signing signifies you are aware of and in accordance with the {\bf Academic Honor Code of Georgia Tech}.
\item Calculators and cell phones are NOT allowed.
\end{itemize}

\section*{Note} This is an object-oriented programming test.  Java is the required language.  Java is case-sensitive.  DO NOT WRITE IN ALL CAPS.  A Java program in all caps will not compile.  Good variable names and style are required.  Comments are not required.

\vfill


% Points Table
%\begin{center}
\addpoints
%\gradetable[h][pages]
%\end{center}

% Points Table
\begin{center}
\renewcommand{\arraystretch}{2}
\begin{tabularx}{\textwidth}{|l|c|X|X|X|}
        \hline
Question & Points per Page & Points Lost & Points Earned & Graded By \\
\hline
Page 1 & \pointsonpage{1} & - & =  &\\
\hline
Page 2 & \pointsonpage{2} & - & =  &\\
\hline
Page 3 & \pointsonpage{3} & - & =  &\\
\hline
Page 4 & \pointsonpage{4} & - & =  &\\
\hline
Page 5 & \pointsonpage{5} & - & =  &\\
\hline
Page 6 & \pointsonpage{6} & - & =  &\\
\hline
TOTAL & \numpoints & - & =  & \\
\hline
\end{tabularx}
\end{center}

\newpage

%\normalsize

\pointsinmargin
\bracketedpoints

\marginpointname{}
%%%%%%%%%%%%%%%%%%%%%%%%%%%%%%%%%%%%%%%%%%%%%%%%%%%%%%%%%%%%%%%%%%%%%%%%%%%%

\begin{questions}

\question {\bf Multiple Choice} Circle the letter of the best answer.

\begin{parts}

\part[2] Given the following code:
\begin{lstlisting}[language=Java]
        ArrayList tasks = new ArrayList(10);
        tasks.add("Eat");
        tasks.add("Sleep");
        tasks.add("Code");
\end{lstlisting}

How many more items can be added to {\tt tasks}?
\begin{choices}
\choice 0
\choice 7
\correctchoice as many as memory will allow, essentially unlimited
\choice None of the above.
\end{choices}

\part[2] What is true about the following code:
\begin{lstlisting}[language=Java]
ArrayList<Integer> myInts = new ArrayList<Integer>();
myInts.add(2);
myInts.add(3);
\end{lstlisting}

\begin{choices}
\choice It will not compile becuase no capacity was given in the {\tt ArrayList} constructor;
\choice It will not compile because you can only add reference variables to collections.
\correctchoice The {\tt int} arguments to {\tt add} will be auto-boxed to {\tt Integer}s.
\choice None of the above.
\end{choices}


\part[2] After the following lines execute:

\begin{lstlisting}[language=Java]
  Map<String, String> capitals = new HashMap<>();
  capitals.put("Georgia", "Atlanta");
  capitals.put("Alabama", "Montgomery");
  capitals.put("Florida", "Tallahassee");
  capitals.put("Georgia", "Valdosta");
\end{lstlisting}

What would {\tt capitals.size()} return?

\begin{choices}
\correctchoice 3
\choice 4
\choice 8
\end{choices}

\part[2] After the following lines execute:

\begin{lstlisting}[language=Java]
  Map<String, String> capitals = new HashMap<>();
  capitals.put("Georgia", "Atlanta");
  capitals.put("Alabama", "Montgomery");
  capitals.put("Florida", "Tallahassee");
  capitals.put("Tennessee", "Atlanta")
\end{lstlisting}

What would {\tt capitals.size()} return?

\begin{choices}
\choice 3
\correctchoice 4
\choice 8
\end{choices}

\end{parts}

\newpage

\question {\bf Multiple Choice} Circle the letter of the best answer.

\begin{parts}

\part[2] Given the following classes and variable initializations:

\begin{lstlisting}[language=Java]
public class A implements Comparable<A> { ... }
public class B extends A { ... }
public class MyComparator implements Comparator<A> { ... }
List<A> aList = ... ;
List<B> bList = ... ;
List<MyComparator> comparatorList = ... ;
\end{lstlisting}

and the signature of Collections.sort():

\begin{lstlisting}[language=Java]
public static <T extends Comparable<? super T>> void sort(List<T> list)
\end{lstlisting}

Which of the following lines will compile?

\begin{choices}
\choice {\tt Collections.sort(aList)}
\choice {\tt Collections.sort(bList)}
\choice {\tt Collections.sort(comparatorList)}
\correctchoice A and B above
\choice All of the above
\end{choices}

\part[2] Given the classes:

\begin{verbatim}
public interface Employee
public class SalariedEmployee implements Employee
public class HourlyEmployee implements Employee
public class SummerIntern extends HourlyEmployee
public class Company<T extends Employee>
\end{verbatim}
\vspace{.1in}

Which of the following lines will {\bf not} compile?

\begin{choices}
\choice {\tt Company<SalariedEmployee> company = new Company<>();}
\choice {\tt Company<HourlyEmployee> company = new Company<>();}
\choice {\tt Company<SummerIntern> company = new Company<>();}
\correctchoice All of the lines above will compile.
\end{choices}


\part[2] Consider the following class:

\begin{lstlisting}[language=Java]
public class MyCollection {
    ...
    public Iterator() iterator() { ... }
}
\end{lstlisting}

What is true about the following code?

\begin{lstlisting}[language=Java]
MyCollection mc = new MyCollection();
mc.add(...);
...
for (Object element: mc) {
     ...
}
\end{lstlisting}

\begin{choices}
\choice It will compile and run without error.
\choice It will compile but produce a runtime error.
\correctchoice It will not compile.
\end{choices}

\end{parts}


\newpage

\question {\bf Multiple Choice} Circle the letter of the best answer. Assume {\tt Trooper} is defined as follows:
\begin{lstlisting}[language=Java]
public class Trooper {
    private String name;
    private boolean mustached;
    public Trooper(String name, boolean hasMustache) {
        this.name = name; this.mustached = hasMustache;
    }
    public String getName() { return name; }
    public boolean hasMustache() { return mustached; }
    public boolean equals(Object other) {
        if (this == other) return true;
        if (null == other || !(other instanceof Trooper)) return false;
        Trooper that = (Trooper) other;
        return this.name.equals(that.name) && this.mustached == that.mustached;
    }
    public int hashCode() { return super.hashCode(); }
}
\end{lstlisting}
And the following has been executed in the same scope as the code in the questions below:
\begin{lstlisting}[language=Java]
  ArrayList<Trooper> troopers = new ArrayList<>();
  troopers.add(new Trooper("Farva", true));
  troopers.add(new Trooper("Farva", true));
  troopers.add(new Trooper("Rabbit", false));
  troopers.add(new Trooper("Mac", true));
\end{lstlisting}

\begin{parts}

\part[2] What would be the result of the statement   {\tt Collections.sort(troopers)}?

\begin{choices}
\correctchoice The code will not compile.
\choice {\tt troopers} will be sorted in order by name.
\choice {\tt troopers} will be sorted in order by mustache, then name.
\choice {\tt troopers} will not have any duplicate elements.
\end{choices}

\part[2] After executing the statement {\tt Set<Trooper> trooperSet = new HashSet<>(troopers)}, what would be the value of {\tt trooperSet.contains(new Trooper("Mac", true))}?

\begin{choices}
\choice The code will not compile.
\choice {\tt true}
\correctchoice {\tt false}
\choice {\tt void}
\end{choices}

\part[2] Given the definitions of {\tt troopers} and {\tt trooperSet} above, what would {\tt troopers.size()} return?

\begin{choices}
\choice {\tt true}
\choice {\tt false}
\choice 3
\correctchoice 4
\end{choices}

\part[2] After the statement {\tt Set<String> stringSet = new HashSet<>(Arrays.asList("meow", "meow"))} executes, what would be the value of {\tt stringSet.size()}?

\begin{choices}
\choice {\tt true}
\choice {\tt false}
\correctchoice 1
\choice 2
\end{choices}



\end{parts}

\newpage

\question {\bf Short Answer}
\begin{parts}

\part[5] Given the definition of {\tt Trooper} and the {\tt ArrayList<Trooper> troopers} in the previous question, write a {\bf single statement} that sorts {\tt troopers} by mustache, then name using {\tt Collections}'s\\ {\tt public static <T> void sort(List<T> list, Comparator<? super T> c)} method.  Assume that you have no helper objects to use.  All the comparison logic must be in this statement.

\ifprintanswers
\begin{lstlisting}[language=Java]
        Collections.sort(troopers, new Comparator<Trooper>() {
                public int compare(Trooper a, Trooper b) {
                    if (a.hasMustache() && !b.hasMustache()) {
                        return 1;
                    } else if (b.hasMustache() && !a.hasMustache()) {
                        return -1;
                    } else {
                        return a.getName().compareTo(b.getName());
                    }
                }
            });
\end{lstlisting}
\else
\vspace{2.5in}
\fi

\part[5] Write a single statement that assigns to a variable named {\tt byMustacheThenName} an object that implements {\tt Comparator<Trooper>} using the methods\\
\begin{lstlisting}[language=Java]
<U extends Comparable<? super U>> Comparator<T>
        comparing(Function<? super T,? extends U> keyExtractor)

<U extends Comparable<? super U>> Comparator<T>
        thenComparing(Function<? super T,? extends U> keyExtractor)
\end{lstlisting}
from {\tt Comparator} and method references for {\tt Trooper}'s {\tt hasMustache()} and {\tt getName()} methods.

\ifprintanswers
\begin{lstlisting}[language=Java]
Comparator<Trooper> byMustacheThenName = Comparator
    .comparing(Trooper::hasMustache)
    .thenComparing(Trooper::getName);
\end{lstlisting}
\else
\vspace{2in}
\fi





\part[5] Following from the previous part, re-write the call to {\tt Collections}'s {\tt public static <T> void sort(List<T> list, Comparator<? super T> c)} from above using the helper object .

\ifprintanswers
\begin{lstlisting}[language=Java]
Collections.sort(troopers, byMustacheThenName);
\end{lstlisting}
\else
\vspace{1.5in}
\fi


\end{parts}

\newpage

\question {\bf Short Answer}
\begin{parts}


\part[5] Write a line of code that instantiates an {\tt ArrayList} object named {\tt labels} that can hold {\tt Label} elements (and only {\tt Label}s) with an initial capacity of 20 and does not produce any compiler errors or warnings. Assume necessary imports.

\begin{solution}[.5in]
\verb@ArrayList<Label> labels = new ArrayList<>(20);@
\end{solution}

\part[5] Continuing from the previous question, write a for-each loop that prints to the console the the text of each {\tt Label} in the {\tt labels} that is not disabled.  Assume {\tt Label} has {\tt String getText()} and {\tt boolean isDisabled()} methods.

\begin{solution}[1.5in]
\begin{verbatim}
for (Label label: labels) {
    if (!label.isDisabled()) {
        System.out.println(label.getText());
    }
}
\end{verbatim}
\end{solution}


\part[5] Write a stream pipeline that does the same thing as in the previous part above.

\begin{solution}[1.5in]
\begin{verbatim}
labels.stream()
    .filter(e -> !isDisabled())
    .forEach(e -> System.out.println(e.getText()))
\end{verbatim}
\end{solution}



\end{parts}

\newpage

\question[10] Fill in the {\tt hasNext()} and {\tt next()} methods in {\tt DynamicArrayIterator}.  If {\tt hasNext()} returns {\tt false}, a call to {\tt next()} should throw a {\tt NoSuchElementException}, which as a no-arg constructor.

\begin{verbatim}
import java.util.Arrays;
import java.util.Iterator;

public class DynamicArray<E> implements Iterable<E> {
    private class DynamicArrayIterator implements Iterator<E> {
        private int cursor = 0;

        public boolean hasNext() {
\end{verbatim}
\ifprintanswers
\begin{lstlisting}
            return cursor <= lastIndex;
\end{lstlisting}
\else
\vspace{.3in}
\fi
\begin{verbatim}
        }

        public E next() {
\end{verbatim}
\ifprintanswers
\begin{lstlisting}
            if (!hasNext()) { throw new NoSuchElementException(); }
            E answer = get(cursor);
            cursor++;
            return answer;
\end{lstlisting}
\else
\vspace{.4in}
\fi
\begin{verbatim}
        }
        public void remove() { throw new UnsupportedOperationException(); }
    }

    private Object[] elements;
    private int lastIndex;

    public DynamicArray() { this(10); }

    public DynamicArray(int capacity) {
        this.elements = new Object[capacity];
        lastIndex = -1;
    }
    public Iterator<E> iterator() {
        return new DynamicArrayIterator();
    }
    public void add(E item) {
        if (lastIndex == elements.length - 1) {
            int newCapacity = elements.length * 2;
            elements = Arrays.copyOf(elements, newCapacity);
        }
        elements[++lastIndex] = item;
    }
    public E get(int index) {
        if ((index < 0) || (index > lastIndex)) {
            throw new IndexOutOfBoundsException(new Integer(index).toString());
        }
        return (E) elements[index];
    }
    public int size() { return lastIndex + 1; }
}
\end{verbatim}



\end{questions}

\end{document}
